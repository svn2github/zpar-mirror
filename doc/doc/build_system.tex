\documentclass[12pt]{article}
\usepackage[latin1]{inputenc}
\usepackage{graphicx}
\usepackage{makeidx}
\usepackage{hyperref}
\usepackage{CJK}
\usepackage{times}
\makeindex
\title{Introduction for ZPar build system}
\begin{document}
\begin{CJK}{GBK}{song}
\maketitle

\section{Overview}
\label{sec:overview}
ZPar is built using the \href{http://www.gnu.org/software/make/}{make} build system
by providing a set of predefined rules and compilation actions.
You may easily customize ZPar or contribute third-party code
by adapting the Makefiles.

\section{How to compile ZPar}
Suppose that ZPar has been downloaded to the directory \textit{zpar}. 
To make a POS tagging system for English, 
type \textit{make english.postagger}. 
This will create a directory \textit{zpar/dist/english.postagger}, 
in which there are two files: \textit{train} and \textit{tagger}. 
The file \textit{train} is used to train a tagging model,
and the file \textit{tagger} is used to tag new texts using a trained parsing model.
\\
\\
Parsers and POS taggers for other languages can be built in similar ways,
please refer to specific manuals for detailed instructions.

\section{How to change implementations}
\label{sec:how-change-impl}

ZPar provides various implementations for parsers and POS taggers for all the supported languages.
To change the implementations, simply edit the different \texttt{*\_IMPL} macro in \textit{zpar/Makefile}.
For example, to use TWeb as POS tagger for generic language, just make the following change to the \textit{zpar/Makefile}:
\\
\\
\hspace{3cm}\texttt{GENERIC_TAGGER_IMPL = tweb}
\\
\\
Then build the generic POS tagger as usual.
\\
\\
To examine which implementations are support for a specific task,
please consult into the sub-folders under:
\\
\\
\texttt{zpar/src/\textit{<LANGUAGE>}/\textit{<TASK>}/implementations/}
\\
\\
The name of the sub-folder is also that of the implementation.

\section{How to contribute code}
\label{sec:customizing-zpar}

You may add a new implementation for a specific task
by creating a sub-folder as described in Section~\nameref{sec:how-change-impl}.
\\
\\
ZPar provides a framework of transition based system
and requires all the implementations compatible with
ZPar APIs and source/object file naming conventions.
You should look into an existing implementation for details.
\\
\\
If the custom implementation is based on the framework
of transition based system of ZPar, 
there's nothing special to do.
Simple change the different \texttt{*\_IMPL} macro in \textit{zpar/Makefile}
into the name of the custom implementation and build as usual.
The build system will automatically detect the implementation,
compile it and link it into ZPar.
\\
\\
Furthermore, ZPar also support integrating implementations
that do not use the framework of transition based system.
In such situation, make a copy the corresponding Makefile template
for the task and language 
in \textit{zpar/Makefile.d/} folder:
\\
\\
\hspace{3cm}\texttt{cp Makefile.\textit{<LANGUAGE>}.\textit{<TASK>} \textbackslash \\
\hspace{3cm}\quad Makefile.\textit{<LANGUAGE>}.\textit{<TASK>}.\textit{<IMPL_NAME>}}
\\
\\
And change the rules in it to compile the custom implementation.
Note that the object file name conventions must be strictly followed
to link the new implementation into ZPar.
You should link all the required objects of your own into one with proper name.
For details, \textit{zpar/Makefile.d/Makefile.ge.postagger.tweb} can be taken as an example.

\end{CJK}
\end{document}
