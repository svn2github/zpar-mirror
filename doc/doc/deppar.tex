\documentclass[12pt]{article}
\usepackage[latin1]{inputenc}
\usepackage{graphicx}
\usepackage{makeidx}
\usepackage{hyperref}
\usepackage{CJK}
\usepackage{times}
\makeindex
\title{Dependency Parsing}
\begin{document}
\begin{CJK}{GBK}{song}
\maketitle

\section{How to compile}
Suppose that Zpar has been downloaded to the directory \textit{zpar}. To make a dependency parsing system for English, type \textit{make english.depparser}. This will create a directory \textit{zpar/dist/english.depparser}, in which there are two files: \textit{train} and \textit{depparser}. The file \textit{train} is used to train a parsing model,and the file \textit{depparser} is used to parse new texts using a trained parsing model. Similarly, we can make a dependency parsing system for Chinese by typing \textit{make chinese.depparser}. The \textit{train} and \textit{depparser} files are created under the directory of \textit{zpar/dist/chinese.depparser}.
\section{Format of inputs and outputs}
The input file to the \textit{train} file contains a set of parse trees. An example parse tree is as follows:
\\
\\
%\begin{tabular}{l}
\hspace{3cm} Ms. \hspace{2cm} NNP \hspace{2cm} 1 \\
\hspace{3cm} Haag \hspace{1.6cm} NNP \hspace{2cm} 2 \\
\hspace{3cm} plays \hspace{1.2cm} VBZ \hspace{2cm} -1 \\
\hspace{3cm} Elianti \hspace{0.8cm} NNP \hspace{2cm} 2 \\
\hspace{3cm} . \hspace{2cm} . \hspace{2cm} 2 \\
Here the first column represents the words of the sentence; the second column contains POS tags of the words; the third column represents the indices of the heads of the words. Indices start from 0. For example, the head index of the word \textit{Ms.} is 1, which means its head is \textit{Haag}. Note that, in each line tab characters are used to separate a word, a POS tag, and an index.
\\
\\
The input file to the \textit{depparser} contain POS tagged sentences. The formats for English and Chinese are different.
\\
\\
\hspace{2cm} \textbf{English}: \\
\hspace{3cm} Ms./NNP Haag/NNP plays/VBZ Elianti/NNP\\
\hspace{2cm}\textbf{Chinese}: \\
\hspace{3cm} ZPar$\_$NR ����$\_$MD ����$\_$VV ����$\_$NN ��$\_$CC Ӣ��$\_$NN \\
\\
For Chinese, inputs to both \textit{train} and \textit{depparser} must be encoded in \textit{utf8}.
\section{How to train a model}
To train a model, use
\\
\\
\hspace{3cm} zpar/dist/english.depparser/train $<$train-file$>$ $<$model-file$>$ $<$number of iterations$>$ \\
\\
For example, using the English \href{eng_dep_files/train.txt}{train file}, you can train a  model by
\\
\\
\hspace{3cm} zpar/dist/english.depparser/train train.txt model 1 \\
\\
After training is completed, a new file \textit{model} will be created in the current directory, which can be used to parse POS-tagged sentences. The above command performs training with one iteration using the training file.
\\
\\
The commands for training Chinese parsing models are the same. For example, using the Chinese \href{chn_dep_files/train.txt}{train file}, you can train a model by
\\
\\
\hspace{3cm} zpar/dist/chinese.depparser/train train.txt model 1 \\
\section{How to parse new texts}
To apply an existing model to parse new texts, use
\\
\\
\hspace{3cm} zpar/dist/english.depparser/depparser $<$model$>$ $<$input-file$>$ $<$output-file$>$
\\
\\
For example, using the model we just trained, we can parse \href{eng_dep_files/input.txt}{an example input} by
\\
\\
\hspace{3cm} zpar/dist/english.depparser/depparser model input.txt output.txt
\\
\\
The output file contains automatically parsed trees. The commands for parsing Chinese texts are the same. See an example of \href{chn_dep_files/input.txt}{Chinese input file}.
\section{Outputs and evaluation}
In order to evaluate the quality of the outputs, we can manually specify the gold parse trees of a sample, and compare the outputs with the correct sample.
\\
Manually specified parse trees of the input file are given in \href{eng_dep_files/reference.txt}{reference file} (find a Chinese reference file \href{chn_dep_files/reference.txt}{here}).  Here is a \href{eng_dep_files/evaluate.py}{Python script} that performs automatic evaluation.
\\
Using the above \textit{output.txt} and \textit{reference.txt}, we can evaluate the accuracies by typing
\\
\\
\hspace{3cm} python evaluate.py output.txt reference.txt
\\
\\
You can find the precision, recall, and f-score here. See the explanation of these measures on \href{http://en.wikipedia.org/wiki/Precision_and_recall}{Wikipedia}.
\section{How to tune the performance of a system}
The performance of the system after one training iteration may not be optimal. You can try training a model for another few iterations, after each you compare the performance. You can choose the model that gives the highest f-score on your test data. We conventionally call this test file the development test data, because you develop a parsing model using this. Here is a \href{eng_dep_files/test.sh}{a shell script} that automatically trains the parser for 30 iterations, and after the $i$th iteration, stores the model file to model.$i$. You can compare the f-score of all 30 iterations and choose model.$k$, which gives the best f-score, as the final model. In this file, this is a variable called \textit{zpar}. You need to set this variable to the relative directory of \textit{zpar/dist/english.depparsr} or \textit{zpar/dist/chinese.depparser}.
\begin{thebibliography}{99}
\bibitem{bib-1}
Yue Zhang and Stephen Clark. 2008. A Tale of Two Parsers: Investigating and Combining Graph-based And transition-based Dependency Parsing Using Beam-search. In {\em Proc. of EMNLP} 2008, pages 562-571.
\bibitem{bib-2}
Yue Zhang and Joakim Nivre. Transition-based Dependency Parsing with Rich Non-local Features. In {\em Proc. of ACL} 2011, pages 188-193.
\end{thebibliography}
\end{CJK}
\end{document}
