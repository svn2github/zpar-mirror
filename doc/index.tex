\documentclass[12pt]{article}
\usepackage[latin1]{inputenc}
\usepackage{graphicx}
\usepackage{makeidx}
\usepackage{hyperref}
\makeindex
\title{User Manual of ZPar}
\author{Yue Zhang \\
frcchang@gmail.com
}
\begin{document}
\maketitle

\section{Overview}

ZPar is a statistical natural language parser, which performs syntactic analysis tasks including word segmentation, part-of-speech tagging and parsing. ZPar supports multiple languages and multiple grammar formalisms. ZPar has been most heavily developed for Chinese and English, while it provides generic support for other languages. A Romanian model has been trained for ZPar 0.2, for example. ZPar currently supports context free grammar (CFG), dependency grammar and combinatory categorial grammar (CCG).

%\textit{ZPar} is a statistical natural language parser, which performs syntactic analysis tasks including word segmentation, part-of-speech tagging and parsing. \textit{ZPar} supports multiple languages and multiple grammar formalisms. \textit{ZPar} currently supports context free grammar (CFG), dependency grammar and combinatory categorial grammar (CCG).

\section{System Requirements}
The ZPar software requires the following basic system configuration
\begin{itemize}
\item Linux or Mac
\item GCC
\item 256MB of RAM minimum
\item At least 500MB of hard disk space
\end{itemize}

\section{Download and Installation}

Download the latest tar ball from \href{http://sourceforge.net/projects/zpar/}{sourceforge} and move the tar ball to your work space.
\\
You can use ZPar by referring to \href{doc/qs.html}{quick start}, or follow detailed instructions in individual modules for installation and usage.
\begin{itemize}
\item \href{doc/segmentor.html}{Chinese word segmentation}
\item \href{doc/joint_seg_tag.html}{Chinese joint segmentation and POS tagging}
\item \href{doc/eng_tagger.html}{English POS tagging}
\item \href{doc/deppar.html}{dependency parsing}
\item  \href{doc/conparser.html}{phrase-structure parsing}
%\item \href{doc/ccg.html}{combinatory categorial grammar}
\end{itemize}

\section{License}

The software source is under GPL (v.3), and a separate commercial license
issued by Oxford University for non-opensource. Various models available for
download were trained from different text resources, which may require further
licenses.

\begin{thebibliography}{99}
\bibitem{bib-1}
Yue Zhang and Stephen Clark. 2011. Syntactic Processing Using the Generalized Perceptron and Beam Search. \textit{Computational Linguistics}, 37(1):105-151.
%\bibitem{bib-2}
%Yue Zhang and Stephen Clark. 2011. Shift-reduce CCG parsing. In {\em Proc. of ACL 2011}, pages 683-692.
\end{thebibliography}
\end{document}
